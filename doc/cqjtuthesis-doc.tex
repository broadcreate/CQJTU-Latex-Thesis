\documentclass[a4paper,12pt]{ctexart}
\usepackage[top=2.5cm,bottom=2.5cm,left=3cm,right=2.5cm]{geometry}
\usepackage{hyperref}
\usepackage{graphicx}
\usepackage{booktabs}
\usepackage{longtable}
\usepackage{listings}
\usepackage{xcolor}
\usepackage{enumitem}

\hypersetup{
    colorlinks=true,
    linkcolor=blue,
    filecolor=magenta,      
    urlcolor=cyan,
}

\lstset{
    basicstyle=\ttfamily\small,
    breaklines=true,
    frame=single,
    backgroundcolor=\color{gray!10},
    keywordstyle=\color{blue},
    stringstyle=\color{red}
}
% 页眉仅显示节标题文本,避免与页码混淆
\renewcommand{\sectionmark}[1]{\markright{#1}}

\title{\textbf{CQJTU-Latex-Thesis 使用指南}}
\author{CQJTUThesis Development Team}
\date{v1.0.2 (2026-02-11)}

\begin{document}

\pagenumbering{gobble}
\maketitle
\thispagestyle{empty}
\clearpage
\pagenumbering{Roman}
\setcounter{page}{1}
\tableofcontents
\clearpage
\pagenumbering{arabic}
\setcounter{page}{1}

\section{简介}

重庆交通大学学位论文 LaTeX 模板,支持本科、硕士(学术型/专业型)、博士论文。

\begin{itemize}
    \item \textbf{版本}: v1.0.2
    \item \textbf{更新日期}: 2026-02-11
    \item \textbf{编译环境}: XeLaTeX + TeX Live 2020+
\end{itemize}

\textbf{格式要求参考}:
\begin{itemize}
    \item 本科论文:《重庆交通大学本科毕业论文(设计)模板》(2024年版)
    \item 研究生论文:《重庆交通大学研究生学位论文格式要求》(2013年修订版)
\end{itemize}

\section{快速开始}

\subsection{本科毕业论文}

\textbf{使用示例文件}: \texttt{examples/cqjtuthesis-bachelor.tex}

\begin{lstlisting}[language=TeX]
\documentclass{../tex/cqjtuthesis}  % 默认为本科模式
% 或显式指定
\documentclass[degree=bachelor]{../tex/cqjtuthesis}
\end{lstlisting}

\textbf{内容目录}: \texttt{examples/contents/} \\
\textbf{编译命令}: \texttt{cd examples \&\& latexmk -xelatex cqjtuthesis-bachelor.tex}

\subsection{学术型硕士学位论文}

\textbf{使用示例文件}: \texttt{examples/cqjtuthesis-master.tex}

\begin{lstlisting}[language=TeX]
\documentclass[degree=master]{../tex/cqjtuthesis}  % 默认为学术型
% 或显式指定
\documentclass[degree=master, degree-type=academic]{../tex/cqjtuthesis}
\end{lstlisting}

\textbf{内容目录}: \texttt{examples/contents-graduate/} \\
\textbf{编译命令}: \texttt{cd examples \&\& latexmk -xelatex cqjtuthesis-master.tex}

\subsection{专业型硕士学位论文}

\textbf{使用示例文件}: \texttt{examples/cqjtuthesis-professional.tex}

\begin{lstlisting}[language=TeX]
\documentclass[degree=master, degree-type=professional]{../tex/cqjtuthesis}
\end{lstlisting}

\textbf{内容目录}: \texttt{examples/contents-graduate/} \\
\textbf{编译命令}: \texttt{cd examples \&\& latexmk -xelatex cqjtuthesis-professional.tex}

\subsection{博士学位论文}

\textbf{使用示例文件}: \texttt{examples/cqjtuthesis-doctor.tex}

\begin{lstlisting}[language=TeX]
\documentclass[degree=doctor]{../tex/cqjtuthesis}
\end{lstlisting}

\textbf{内容目录}: \texttt{examples/contents-graduate/} \\
\textbf{编译命令}: \texttt{cd examples \&\& latexmk -xelatex cqjtuthesis-doctor.tex}

\section{选项说明}

\subsection{类选项}

\begin{longtable}{llll}
\toprule
选项 & 可选值 & 默认值 & 说明 \\
\midrule
\texttt{degree} & \texttt{bachelor} / \texttt{master} / \texttt{doctor} & \texttt{bachelor} & 学位类型 \\
\texttt{degree-type} & \texttt{academic} / \texttt{professional} & \texttt{academic} & 学位类别 \\
\bottomrule
\end{longtable}

\subsection{校徽选项}

\begin{tabular}{ll}
\toprule
选项 & 说明 \\
\midrule
\texttt{\textbackslash coverlogo\{blue\}} & 蓝色校徽(推荐) \\
\texttt{\textbackslash coverlogo\{red\}} & 红色校徽 \\
\texttt{\textbackslash coverlogo\{legacy\}} & 旧版校徽 \\
\bottomrule
\end{tabular}

\section{主要格式区别}

\subsection{页面设置}

\begin{tabular}{lll}
\toprule
项目 & 本科论文 & 研究生论文 \\
\midrule
纸张 & A4 & A4 \\
上边距 & 2.5cm & 3.0cm \\
下边距 & 2.0cm & 2.5cm \\
左边距 & 3.0cm & 3.0cm \\
右边距 & 2.0cm & 3.0cm \\
装订线 & 无 & 0.5cm \\
\bottomrule
\end{tabular}

\subsection{摘要与目录}

\begin{tabular}{lll}
\toprule
项目 & 本科 & 研究生 \\
\midrule
摘要标题 & “摘 要”(小三号黑体) & “摘  要”(三号黑体,2空格) \\
摘要正文 & 小四号宋体,1.5倍行距 & 小四号宋体,20磅固定行距 \\
\bottomrule
\end{tabular}

\section{封面信息定义}

\subsection{基本信息(所有学位类型)}

\begin{lstlisting}[language=TeX]
\title{论文题目}
\englishtitle{English Title}
\author{你的姓名}
\studentid{你的学号}
\school{你的学院}
\major{你的专业}
\advisor{指导教师姓名 职称}
\completedate{2024年6月}
\coverlogo{blue}
\end{lstlisting}

\subsection{本科生额外字段}

\begin{lstlisting}[language=TeX]
\cohort{20XX}                  % 届别,如2024
\classnum{你的班级}             % 班级号
\end{lstlisting}

\subsection{研究生额外字段}

\begin{lstlisting}[language=TeX]
\clc{U491}                     % 分类号
\securitylevel{保密}            % 密级
\discipline{交通运输工程}       % 一级学科
\subdiscipline{交通信息工程及控制}  % 二级学科
\degreecategory{硕士学位}       % 申请学位类别
\submitdate{2024年5月15日}     % 论文提交日期
\defenddate{2024年5月20日}     % 论文答辩日期
\end{lstlisting}

\section{文件结构}

\begin{lstlisting}
cqjtuthesis/
├── README.md                    # 说明文件
├── tex/                         # TeX运行时文件
│   └── cqjtuthesis.cls         # 模板类文件
├── doc/                         # 文档
│   ├── cqjtuthesis-doc-zh.md   # 中文使用指南
│   └── cqjtuthesis-doc.tex     # PDF文档源码
└── examples/                    # 示例文件
    ├── cqjtuthesis-bachelor.tex
    ├── cqjtuthesis-master.tex
    ├── cqjtuthesis-professional.tex
    ├── cqjtuthesis-doctor.tex
    ├── contents/                # 本科内容
    ├── contents-graduate/       # 研究生内容
    └── figures/                 # 图片文件夹
\end{lstlisting}

\section{使用步骤}

\begin{enumerate}
    \item \textbf{选择正确的模板文件}:根据学位类型选择对应的 \texttt{.tex} 文件。
    \item \textbf{修改封面信息}:修改 \texttt{\textbackslash title}, \texttt{\textbackslash author} 等信息。
    \item \textbf{修改示例内容}:本修改 \texttt{contents/} 或 \texttt{contents-graduate/} 下的源文件。
    \item \textbf{添加参考文献}:在 \texttt{examples/bibliography/cqjtuthesis-refs.bib} 中添加。
    \item \textbf{编译生成PDF}:推荐使用 \texttt{latexmk} 或反复执行 \texttt{xelatex -> bibtex -> xelatex -> xelatex}。
\end{enumerate}

\section{常见问题}

\textbf{Q: 如何切换学位类型?} \\
A: 修改 \texttt{\textbackslash documentclass} 的 \texttt{degree} 选项。

\textbf{Q: 论文题目太长怎么办?} \\
A: 建议精简标题,模板统一使用单行显示。

\textbf{Q: 编译出错怎么办?} \\
A: 检查文件路径、确保安装了中文字体(SimSun, SimHei)、检查参考文献bib文件。

\section{更多信息}

\begin{itemize}
    \item 官方格式要求:请参考学校官网相关模板。
    \item 技术支持:建议使用 TeX Live 2020+ 环境。
\end{itemize}

\end{document}
