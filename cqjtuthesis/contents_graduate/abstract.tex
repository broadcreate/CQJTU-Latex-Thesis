% 研究生论文中文摘要示例
% ==========================================
% 格式说明:
% - 标题"摘要":三号黑体,居中,字间距为两个空格
% - 正文:小四号宋体,首行缩进2字符,行距20磅(固定值)
% - 关键词:小四号黑体"关键词:"+ 小四号宋体正文,用分号分隔
% ==========================================

\begin{cabstract}
本文针对智能交通系统优化问题,提出了一种基于深度学习的交通流量预测与优化方法。通过构建多层神经网络模型,实现了对城市道路网络交通流量的准确预测,并结合强化学习算法优化信号灯控制策略。

研究首先收集了重庆市主城区典型路段的交通流量数据,建立了包含时空特征的交通数据集。在此基础上,设计了融合卷积神经网络(CNN)和长短期记忆网络(LSTM)的混合深度学习模型,用于捕捉交通流的时空演化规律。实验结果表明,该模型在短期交通流预测任务中的平均绝对误差(MAE)降低了23.5\%,均方根误差(RMSE)降低了18.7\%。

进一步,本文将交通信号控制问题建模为马尔可夫决策过程(MDP),采用深度Q网络(DQN)算法学习最优控制策略。仿真实验显示,与传统固定配时方案相比,基于DQN的自适应信号控制使平均车辆延误时间减少了31.2\%,路网通行能力提升了24.6\%。

本研究为智能交通系统的优化提供了新的技术路径,具有重要的理论意义和应用价值。

% 关键词格式说明:
% - "关键词:"三个字用小四号黑体
% - 关键词用小四号宋体,用分号";"分隔,最后一个关键词后不加标点
\ckeywords{智能交通系统;深度学习;交通流量预测;信号灯优化;强化学习}
\end{cabstract}

% 英文摘要示例
% ==========================================
% 格式说明:
% - 标题"ABSTRACT":三号黑体(或Times New Roman),居中
% - 正文:小四号Times New Roman,首行缩进2字符,行距20磅
% - 关键词:小四号加粗"KEY WORDS:"+ 小四号Times New Roman
% ==========================================

\begin{eabstract}
This thesis addresses the optimization problem of intelligent transportation systems by proposing a deep learning-based traffic flow prediction and optimization method. Through the construction of a multi-layer neural network model, accurate prediction of traffic flow in urban road networks is achieved, and reinforcement learning algorithms are combined to optimize traffic signal control strategies.

The research first collected traffic flow data from typical road sections in the main urban area of Chongqing and established a traffic dataset containing spatiotemporal features. On this basis, a hybrid deep learning model integrating Convolutional Neural Networks (CNN) and Long Short-Term Memory networks (LSTM) was designed to capture the spatiotemporal evolution patterns of traffic flow. Experimental results show that the model reduces the Mean Absolute Error (MAE) by 23.5\% and the Root Mean Square Error (RMSE) by 18.7\% in short-term traffic flow prediction tasks.

Furthermore, this thesis models the traffic signal control problem as a Markov Decision Process (MDP) and employs the Deep Q-Network (DQN) algorithm to learn optimal control strategies. Simulation experiments demonstrate that compared to traditional fixed-time schemes, DQN-based adaptive signal control reduces average vehicle delay by 31.2\% and increases network capacity by 24.6\%.

This research provides a new technical approach for the optimization of intelligent transportation systems, with significant theoretical and practical value.

\ekeywords{Intelligent Transportation System; Deep Learning; Traffic Flow Prediction; Signal Optimization; Reinforcement Learning}
\end{eabstract}
