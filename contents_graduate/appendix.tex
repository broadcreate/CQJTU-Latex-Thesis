% 附录 - 研究生论文(可选)
% ==========================================
% 格式说明:
% - 标题"附录A/B/...":三号黑体,居中
% - 正文:小四号宋体,首行缩进2字符,行距20磅(固定值)
% - 主文件已调用 \appendix,此处直接使用 \chapter
% ==========================================

\chapter{实验数据统计表}

本附录给出了实验中使用的主要数据集统计信息。

\section{交通流量数据集统计}

\begin{table}[htbp]
  \centering
  \caption{交通流量数据集基本信息}
  \begin{tabular}{lcccc}
    \toprule
    数据集 & 采集地点 & 时间跨度 & 数据量 & 采样间隔 \\
    \midrule
    Dataset-1 & 观音桥路口 & 2023.01-2023.06 & 26,280条 & 5分钟 \\
    Dataset-2 & 解放碑路口 & 2023.01-2023.06 & 26,280条 & 5分钟 \\
    Dataset-3 & 沙坪坝路口 & 2023.01-2023.06 & 26,280条 & 5分钟 \\
    \bottomrule
  \end{tabular}
\end{table}

\section{模型参数配置}

\begin{table}[htbp]
  \centering
  \caption{CNN-LSTM模型参数设置}
  \begin{tabular}{ll}
    \toprule
    参数名称 & 参数值 \\
    \midrule
    CNN卷积核大小 & $3\times 3$ \\
    CNN卷积核数量 & 64 \\
    LSTM隐藏层单元数 & 128 \\
    学习率 & 0.001 \\
    批量大小 & 32 \\
    训练轮数 & 100 \\
    优化器 & Adam \\
    \bottomrule
  \end{tabular}
\end{table}

\begin{table}[htbp]
  \centering
  \caption{DQN算法参数设置}
  \begin{tabular}{ll}
    \toprule
    参数名称 & 参数值 \\
    \midrule
    状态空间维度 & 12 \\
    动作空间维度 & 8 \\
    经验回放缓冲区大小 & 10,000 \\
    目标网络更新频率 & 100 \\
    折扣因子$\gamma$ & 0.99 \\
    探索率$\epsilon$ & 0.1 \\
    学习率 & 0.0001 \\
    \bottomrule
  \end{tabular}
\end{table}

\chapter{部分实验结果图表}

\section{训练过程可视化}

本节展示了模型训练过程中损失函数和评价指标的变化曲线。

% 注:实际使用时需要插入相应的图片
% \begin{figure}[htbp]
%   \centering
%   \includegraphics[width=0.8\textwidth]{figures/training_loss.png}
%   \caption{训练损失曲线}
% \end{figure}

\section{预测结果对比}

% \begin{figure}[htbp]
%   \centering
%   \includegraphics[width=0.8\textwidth]{figures/prediction_comparison.png}
%   \caption{不同方法的预测结果对比}
% \end{figure}
