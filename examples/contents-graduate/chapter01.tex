% 第一章示例 - 研究生论文
% ==========================================
% 格式说明:
% - 章标题:三号黑体,居中,"第X章"格式
% - 一级节标题:小三号黑体,左对齐
% - 二级节标题:四号黑体,左对齐
% - 正文:小四号宋体,首行缩进2字符,行距20磅(固定值)
% - 图表标题:五号黑体
% ==========================================

\chapter{绪论}

% 一级节标题格式说明:小三号黑体,左对齐
\section{研究背景与意义}

随着城市化进程的加速,交通拥堵已成为制约城市可持续发展的重要瓶颈。据统计,我国大中型城市高峰时段平均车速已降至20km/h以下,交通拥堵造成的经济损失每年超过数千亿元。智能交通系统(Intelligent Transportation System, ITS)作为解决交通问题的重要手段,受到了学术界和工业界的广泛关注\cite{example}。

% 二级节标题格式说明:四号黑体,左对齐
\subsection{智能交通系统发展现状}

智能交通系统是将先进的信息技术、数据通信传输技术、电子传感技术、控制技术及计算机技术等有效地集成运用于整个地面交通管理系统而建立的一种在大范围内、全方位发挥作用的实时、准确、高效的综合交通运输管理系统。

当前,智能交通系统主要包括以下几个方面:

\begin{enumerate}
  \item \textbf{交通信息采集系统}:通过各类传感器实时采集道路交通信息
  \item \textbf{交通信号控制系统}:根据实时交通状况优化信号灯配时
  \item \textbf{交通诱导系统}:为出行者提供实时路况和最优路径建议
  \item \textbf{公共交通管理系统}:提高公共交通服务质量和运营效率
\end{enumerate}

\subsection{深度学习在交通领域的应用}

近年来,深度学习技术在计算机视觉、自然语言处理等领域取得了突破性进展,也逐渐被应用于交通领域。深度学习具有强大的特征学习和模式识别能力,能够从海量交通数据中自动提取有效特征,为交通流预测、交通状态识别、交通事件检测等任务提供了新的解决思路。

\section{国内外研究现状}

\subsection{交通流预测研究现状}

交通流预测是智能交通系统的核心问题之一。传统方法主要包括:

\begin{itemize}
  \item \textbf{统计模型}:如ARIMA模型、卡尔曼滤波等
  \item \textbf{机器学习方法}:如支持向量机(SVM)、K近邻(KNN)等
  \item \textbf{深度学习方法}:如循环神经网络(RNN)、长短期记忆网络(LSTM)等
\end{itemize}

% 表格示例
% 格式说明:表题为五号黑体,表内文字为小五号宋体
\begin{table}[htbp]
  \centering
  \caption{不同交通流预测方法对比}
  \begin{tabular}{cccc}
    \toprule
    方法类别 & 代表方法 & 预测精度 & 计算复杂度 \\
    \midrule
    统计模型 & ARIMA & 中等 & 低 \\
    机器学习 & SVM & 较高 & 中等 \\
    深度学习 & LSTM & 高 & 高 \\
    \bottomrule
  \end{tabular}
\end{table}

\subsection{交通信号控制研究现状}

交通信号控制是缓解交通拥堵的重要手段。当前主要的控制策略包括:

\begin{enumerate}
  \item \textbf{固定配时控制}:根据历史数据设定固定的信号配时方案
  \item \textbf{感应控制}:根据实时检测到的车辆信息调整配时
  \item \textbf{自适应控制}:利用优化算法动态调整信号配时
  \item \textbf{智能控制}:基于人工智能技术的自主学习控制
\end{enumerate}

\section{论文主要研究内容}

本文的主要研究内容包括:

\begin{enumerate}
  \item 构建融合时空特征的交通流预测模型
  \item 设计基于深度强化学习的自适应信号控制算法
  \item 搭建仿真平台验证所提方法的有效性
  \item 分析不同场景下的优化效果和鲁棒性
\end{enumerate}

\section{论文组织结构}

本文共分为六章,各章节内容安排如下:

第1章为绪论,介绍研究背景、意义和国内外研究现状。

第2章为相关理论基础,介绍深度学习和强化学习的基本原理。

第3章为交通流预测模型设计,详细阐述模型结构和训练方法。

第4章为交通信号控制算法设计,提出基于DQN的自适应控制策略。

第5章为实验与分析,验证所提方法的有效性。

第6章为总结与展望,总结全文工作并展望未来研究方向。
