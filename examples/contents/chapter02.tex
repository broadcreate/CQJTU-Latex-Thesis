% !TeX root = ../main.tex

\chapter{建议及公式的书写规则}

本章介绍论文中建议、公式等内容的书写规范。

\section{建议内容说明}

建议类论文的大体构成如下:

\subsection{基于技术研发性的课题}

内容需包括背景,要解决的实际现实问题,解决方法,解决所需工具,论据。图、表,实验数据,成果应用等。

\subsection{基于数据分析性的课题}

内容需包括数据背景,要解决的实际现实问题,解决所采用的工具和方法,论据、图、表,实验数据,分析成果应用。

\subsection{基于理论研究性的课题}

内容包括研究背景、前人研究理论梳理、研究方法、研究论据,结论等。

\section{公式的书写规则}

\subsection{一般公式}

公式应另起一行书写,并居中排列。公式序号应用圆括号括起,靠公式右侧标注。

简单的公式可在正文中书写,如:$a^2 + b^2 = c^2$。

\subsection{多行公式}

较长的公式,最好在等号或运算符号处转换,转换处用“=”或运算符号标示,示例如下:

\begin{align}
  (a + b)^2 &= a^2 + 2ab + b^2 \notag \\
            &= a^2 + b^2 + 2ab \label{eq:multiline}
\end{align}

\subsection{公式组}

对于多个相关的公式,可以使用公式组:

\begin{subequations}\label{eq:group}
\begin{align}
  x &= r\cos\theta \\
  y &= r\sin\theta
\end{align}
\end{subequations}

引用公式组时为:公式\eqref{eq:group}。

\section{常用数学符号}

\subsection{希腊字母}

常用的希腊字母包括:$\alpha$, $\beta$, $\gamma$, $\delta$, $\epsilon$, $\theta$, $\lambda$, $\mu$, $\pi$, $\sigma$, $\omega$ 等。

\subsection{数学运算符}

\begin{itemize}
  \item 求和:$\sum_{i=1}^{n} x_i$
  \item 连乘:$\prod_{i=1}^{n} x_i$
  \item 积分:$\int_{a}^{b} f(x)\,dx$
  \item 极限:$\lim_{x \to \infty} f(x)$
  \item 偏导数:$\frac{\partial f}{\partial x}$
\end{itemize}

\subsection{矩阵和向量}

矩阵示例:

\begin{equation}
  \mathbf{A} = \begin{bmatrix}
    a_{11} & a_{12} & a_{13} \\
    a_{21} & a_{22} & a_{23} \\
    a_{31} & a_{32} & a_{33}
  \end{bmatrix}
\end{equation}

向量示例:$\mathbf{v} = (v_1, v_2, v_3)^T$。
