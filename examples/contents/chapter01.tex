% !TeX root = ../main.tex

\chapter{正文格式说明}

本章介绍论文正文的格式要求和排版样式。

\section{正文格式基本要求}

\subsection{论文正文基本要求}

论文正文包括:绪论(或前言、序言)、论文主体及结论。

\begin{itemize}
  \item 页面:A4纸,单面打印或双面打印。
  \item 页边距:上2.5cm、下2.0cm、左3.0cm、右2.0cm、装订线0cm。
  \item 页眉:每页页眉居中为“重庆交通大学本科毕业论文(设计)~题目”,宋体,小5号。页眉从正文开始添加。
  \item 正文字体:宋体、小4号、1.5倍行距。
\end{itemize}

\subsection{正文撰写的内容和顺序}

一般由十个主要部分组成,依次为:

\begin{enumerate}
  \item 封面(见附件模板)
  \item 成绩使用声明和版权授权书
  \item 摘要(Abstract)
  \item 目录
  \item 主要符号对照表(必要时)
  \item 正文
  \item 参考文献
  \item 致谢
  \item 附录(必要时)
  \item 外文译文及原文复印件(装订于最后)
\end{enumerate}

\section{正文主体部分}

\subsection{章节标题}

章标题按照“第1章”、“第2章”……的格式编写,采用三号黑体,居中。

节标题按照“1.1”、“1.2”……的格式编写,采用四号黑体,左对齐。

小节标题按照“1.1.1”、“1.1.2”……的格式编写,采用小四号黑体,左对齐。

\subsection{引用和注释}

正文中引用他人的成果,插图,表格数据等,都必须注明来源。正文中引用参考文献时,应在引用处标注右上角标,如:\cite{example}。

\subsection{量和单位}

要严格执行GB 3100~3102:93有关量和单位的规定。单位名称和符号的书写方式,一律采用国际通用符号。

\section{图表和公式}

\subsection{插图}

图应有图号和图题。图号和图题应置于图下方的居中位置。图题为5号黑体。

示例如图\ref{fig:example}所示。

\begin{figure}[htbp]
  \centering
  \includegraphics[width=0.6\textwidth]{image012.jpg}
  \caption{这是一个示例图}
  \label{fig:example}
\end{figure}

\subsection{表格}

表应有表号和表题。表号和表题应置于表上方的居中位置。表题为5号黑体。

表格推荐使用三线表,示例如表\ref{tab:example}所示。

\begin{table}[htbp]
  \centering
  \caption{三线表示例}
  \label{tab:example}
  \begin{tabular}{ccc}
    \toprule
    项目 & 数值 & 单位 \\
    \midrule
    数据1 & 123.45 & m \\
    数据2 & 678.90 & kg \\
    \bottomrule
  \end{tabular}
\end{table}

\subsection{公式}

公式应另起一行,并居中书写。公式编号用圆括号括起,放在公式右边行末。公式与编号之间不加虚线。

示例如公式\eqref{eq:example}所示:

\begin{equation}
  E = mc^2
  \label{eq:example}
\end{equation}

公式中的变量应使用斜体,向量和矩阵使用粗斜体,常数使用正体。
